\htmlhr
\chapterAndLabel{NonEmpty Checker}{nonempty-checker}

The NonEmpty Checker ensures that a Collection, Map, or an Iterator annotated as
\code{@NonEmpty} contains at least one element.

To run the NonEmpty Checker, supply the
\code{-processor org.checkerframework.checker.nonempty.NonEmptyChecker}
command-line option to javac.

\sectionAndLabel{Type Qualifiers and Hierarchy}{nonempty-hierarchy}
The NonEmpty Checker supports the following type qualifiers:
\begin{description}

    \item[\refqualclass{checker/nonempty/qual}{UnknownNonEmpty}]
    Indicates that the value assigned to the annotated variable is not known to be a non-empty
    collection, map, or iterator. It is the default type
    qualifier, so it should rarely be written by a programmer.

    \item[\refqualclass{checker/nonempty/qual}{NonEmpty}]
    Indicates that a collection, map, or iterator value is non-empty.

    \item[\refqualclass{checker/nonempty/qual}{NonEmptyBottom}]
    It is the bottom type in the NonEmpty type system. Programmers should rarely write this type.

    \item[\refqualclass{checker/nonempty/qual}{PolyNonEmpty}]
    indicates qualifier polymorphism.
    For a description of qualifier polymorphism, see
    Section~\ref{method-qualifier-polymorphism}.

\end{description}

\begin{figure}
    \includeimage{nonempty}{4cm}
    \caption{Type hierarchy for the NonEmpty type system.}
    \label{fig-nonempty-hierarchy}
\end{figure}

Figure~\ref{fig-nonempty-hierarchy} shows the qualifier hierarchy in the NonEmpty type system.
