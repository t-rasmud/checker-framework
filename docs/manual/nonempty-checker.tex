\htmlhr
\chapterAndLabel{NonEmpty Checker}{nonempty-checker}

The NonEmpty Checker ensures that a Collection, Map, or an Iterator annotated as
\code{@NonEmpty} contains atleast one element.

\SectionAndLabel{Type Qualifiers and Hierarchy}
The NonEmpty Checker supports the following type qualifiers:
\begin{description}

    \item[\refqualclass{checker/nonempty/qual}{UnknownNonEmpty}]
    Indicates that the value assigned to the annotated variable is not known to be a non-empty
    collection, map, or iterator. It is the top type qualifier in the NonEmpty type hierarchy. It is also the default type
    qualifier.
    It is used internally by the type system and should never be written by a programmer.

    \item[\refqualclass{checker/nonempty/qual}{NonEmpty}]
    Indicates that the collection, map, or iterator value assigned to the annotated variable is non-empty.

    \item[\refqualclass{checker/nonempty/qual}{NonEmptyBottom}]
    It is the bottom type in the NonEmpty type system. Programmers should rarely write this type.

    \item[\refqualclass{checker/nonempty/qual}{PolyNonEmpty}]
    A polymorphic qualifier for the NonEmpty type system.
    Any method written using \code{@PolyNonEmpty} conceptually has an arbitrary number of
    versions: one in which every instance of \code{@PolyNonEmpty} has been replaced by
    \code{@UnknownNonEmpty}, one in which every instance of \code{@PolyNonEmpty}
    has been replaced by \code{@NonEmptyBottom}, and ones in which every instance of
    \code{@PolyNonEmpty} has been replaced by \code{@NonEmpty}, for every possible
    combination of Collection, Map, or Iterator arguments.

\end{description}
