\htmlhr
\chapterAndLabel{Iteration Checker}{iteration-checker}

The Iteration Checker ensures that
\sunjavadoc{java.base/java/util/Iterator.html\#next()}{Iterator.next()}
does not throw \code{NoSuchElementException}.

To run the Iteration Checker, supply the
\code{-processor org.checkerframework.checker.iteration.IterationChecker}
command-line option to javac.

\sectionAndLabel{Type Qualifiers and Hierarchy}{iteration-hierarchy}
The Iteration Checker supports the following type qualifiers:
\begin{description}

    \item[\refqualclass{checker/iteration/qual}{UnknownHasNext}]
    Indicates either a non-iterator, or an iterator that might or might not have a next element.
    This is the default type qualifier.

    \item[\refqualclass{checker/iteration/qual}{HasNext}]
    An expression whose type has this annotation is an iterator that has a next value --- that is,
    \code{hasNext()} will return true and
    \code{next()} will not throw NoSuchElementException.

\end{description}

\begin{figure}
    \includeimage{iteration}{4cm}
    \caption{Type hierarchy for the Iteration type system.}
    \label{fig-iteration-hierarchy}
\end{figure}

Figure~\ref{fig-iteration-hierarchy} shows the qualifier hierarchy in the Iteration type system.

\sectionAndLabel{Iteration Checker Rules}{iteration-rules}

The Iteration Checker has the NonEmpty Checker as its subchecker (\chapterpageref{nonempty-checker})
and issues a warning at a call to \code{Iterator.next()} if either of the
following conditions is true:

\begin{enumerate}
    \item The receiver of \code{next()} has the type qualifier \code{@UnknownHasNext}.
    \item The receiver of \code{next()} doesn't have the type qualifier \code{@NonEmpty}
    from the NonEmpty type system.
\end{enumerate}

\sectionAndLabel{Examples}{iteration-examples}

\begin{verbatim}
    void test1(List<Integer> a) {
        Iterator<Integer> iter = a.iterator();
        if (iter.hasNext()) {
            iter.next(); // OK
        }
        iter.next();  // error!
    }
\end{verbatim}

In the above example, the first call to \code{next} is safe since it is in
the then branch of the conditional (the postcondition of \code{hasNext} ensures that
its receiver gets the annotation \code{@HasNext} if the results of \code{hasNext} is true).
The second call to \code{next()} in unsafe and the Iterator checker issues a warning.

\begin{verbatim}
    void test2() {
        List<Integer> a = new ArrayList<>();
        a.add(5);
        Iterator<Integer> iter = a.iterator();
        iter.next();  // OK
    }
\end{verbatim}

In the example above, the call to \code{next()} isn't preceded by a call to \code{hasNext()}.
However, the call to \code{add()} ensures that the List \code{a} and subsequently
the Iterator \code{iter} are annotated as \code{@NonEmpty} in the NonEmpty type system.
Therefore, the Iteration Checker correctly type checks (second rule in~\ref{iteration-rules}).





