\htmlhr
\chapterAndLabel{Determinism Checker}{determinism-checker}

A nondeterministic program is one that may produce different output when
it is run twice with the same input on the same machine.
It is desirable for programs to be deterministic.
It is easier to test a deterministic program, because nondeterminism can
lead to flaky tests that sometimes succeed and sometimes fail.
It is easier to reproduce a bug in a deterministic program.
It is easier for a user or programmer to compare two
deterministic executions than two nondeterministic executions.

The Determinism Checker helps you make your programs deterministic or
recognize when they are not.
The Determinism Checker enables a programmer to indicate which
computations should be the same across runs of a single-threaded program,
and then the Determinism Checker verifies that property.

Even single-threaded programs are frequently nondeterministic.
One cause is use of the \<Random>, \<Date>, and \<Time> classes.
Another cause is use of hash tables (including \<HashSet>s and
\<HashMap>s), whose iteration order is nondeterministic across runs.
For example, even if the set contains the same values, the following code
may produce output in different orders:

\begin{Verbatim}
  for (Object x : mySet) {
    System.out.println(x);
  }
\end{Verbatim}

Another common cause of nondeterminism is the operating system scheduler,
for concurrent programs.  The Determinism Checker does not address this
cause of nondeterminism.  Also, the Determinism Checker does not warn
about the possibility that the file system contains different files or
contents on different runs of a program.  As a special case, the system
properties for line and path separators are allowed to vary across
different machines, because many tools (such as diff, editors, etc.)
treat all possible values the same.

To run the Determinism Checker, supply the
\code{-processor DeterminismChecker}
command-line option to javac.
For example:

\begin{alltt}
  javac -processor DeterminismChecker InputFile.java
\end{alltt}

%./bin/javac -cp NonDetTypeChecker/jsr308/checker-framework/checker/build -processor org.checkerframework.checker.determinism.DeterminismChecker NDCheckerTest/src/TestListUnsafe2.java


\sectionAndLabel{Determinism qualifiers}{determinism-qualifiers}

The Determinism type system uses the following type qualifiers (see Figure~\ref{fig-determinism-hierarchy}):
\begin{description}
\item[\refqualclass{checker/determinism/qual}{NonDet}] indicates
  that the expression might have different values in two different executions on the same machine.
\item[\refqualclass{checker/determinism/qual}{OrderNonDet}] indicates that
  a collection, iterator, or array will have the same elements in every execution, but in a
  possibly different order.  \<@OrderNonDet> may only be written on
  collection types.
  An \<@OrderNonDet> Map indicates that its keySet is \<@OrderNonDet>.
\item[\refqualclass{checker/determinism/qual}{Det}] indicates that
  the expression evaluates to the same value (with respect to \<.equals()>) in all
  executions on the same machine; for a collection, iteration also yields the values in the same
  order.
  This is the default qualifier.
\end{description}

Section~\ref{determinism-polymorphism} explains polymorphic qualifiers.

\begin{figure}
  \begin{center}
    \includeimage{determinism}{4cm}
  \end{center}
\caption{Determinism type hierarchy}
\label{fig-determinism-hierarchy}
\end{figure}

\sectionAndLabel{Declaration annotations}{determinism-declaration-annotations}

\begin{description}
\item[\refqualclass{checker/determinism/qual}{CollectionType}] is a type declaration annotation used
  to indicate that the class (and all of its subclasses) implements a collection. Only types with this
  annotation and arrays may be annotated with \<@OrderNonDet>.
\end{description}

\sectionAndLabel{Valid collection and array types}{determinism-collection-types}

The (determinism) type of a collection must be a supertype or equal to
the type of the type parameter (see Figure~\ref{fig-determinism-collections}).

Some examples of valid types are:
\begin{itemize}
    \item \code{@NonDet\ \ \ \ \ \ List<@NonDet\ \ \ \ \ \ Integer>}
    \item \code{@NonDet\ \ \ \ \ \ List<@DetDet\ \ \ \ \ \ \ \ \ Integer>}
    \item \code{@OrderNonDet List<@OrderNonDet Set<...>\relax >}
    \item \code{@OrderNonDet List<@Det String>}
    \item \code{@Det\ \ \ \ \ \ \ \ \ List<@Det\ \ \ \ \ \ \ \ \ Integer>}
\end{itemize}

These types are invalid:
  \begin{itemize}
    \item \code{@NonDet\ \ \ \ \ \ List<@PolyDet \ \ \ \ Set<...>\relax >}
    \item \code{@OrderNonDet List<@NonDet\ \ \ \ \ \ Integer>}
    \item \code{@Det\ \ \ \ \ \ \ \ \ List<@NonDet\ \ \ \ \ \ Integer>}
    \item \code{@Det\ \ \ \ \ \ \ \ \ List<@OrderNonDet Set<...>\relax >}
    \item \code{@Det\ \ \ \ \ \ \ \ \ List<T extends @NonDet Object>}
  \end{itemize}

Similarly, the (determinism) type of an array must be a supertype or equal to
the type of the component type (see Figure~\ref{fig-determinism-collections}).

For example, these types are valid:
\begin{itemize}
    \item \code{@NonDet int @NonDet \ \ \ \ []}
    \item \code{@NonDet int @Det \ \ \ \ \ \ \ []}
    \item \code{@Det \ \ \ int @Det \ \ \ \ \ \ \ []}
    \item \code{@Det \ \ \ int @OrderNonDet[] @OrderNonDet[]}
\end{itemize}

These types are invalid:
\begin{itemize}
    \item \code{@Det \ \ \ \ int @NonDet \ \ \ \ []}
    \item \code{@PolyDet int @NonDet \ \ \ \ []}
\end{itemize}

\begin{figure}
  \centering
  \begin{tabular}{|l|l|l|l|l|}
    \cline{3-5}
    \multicolumn{2}{c|}{~}  &  \multicolumn{3}{c|}{Type argument (or array component type)} \\ \cline{3-5}
    \multicolumn{2}{c|}{~}  & NonDet     & OrderNonDet & Det \\ \hline
              & NonDet      &   valid    &  valid    & valid  \\ \cline{2-5}
Collection (or array)   & OrderNonDet &   invalid  &  valid      & valid  \\ \cline{2-5}
              & Det         &   invalid  &  invalid    & valid      \\ \hline
  \end{tabular}
  \caption{Valid Collection (and array) declarations.  The Collection's (or array's) type qualifier
    must be a supertype or equal to the type argument (or array component type).}
  \label{fig-determinism-collections}
\end{figure}

\subsectionAndLabel{Accessing array elements}{determinism-access-array-elements}
When an array of type \<@OrderNonDet> (or \<@PolyDet>) is accessed on the rhs of an expression,
the Determinism Checker cannot guarantee that the same element will be returned in every execution. For instance,
consider an \<@OrderNonDet> array \<a> with elements \code{[5,2,8]}. In one execution,
\<a[0]> might return \<5> and in another execution, it could return \<8>.
So the the type of the array access expression (equivalently, the array element) is
\<@NonDet> (or \<PolyDet("up")>). Example:
\begin{Verbatim}
@Det int @OrderNonDet [] arr;
int val = arr[0];
\end{Verbatim}
In the code above, type of \<arr[0]> gets annotated as \<NonDet>.

Similarly, when a \code{@NonDet int} is used to index into an array,
the Determinism Checker annotates the type of the array access expression as \<@NonDet>. Example:
\begin{Verbatim}
@Det int @Det [] arr;
@NonDet int index;
int val = arr[index];
\end{Verbatim}
In the code above, type of \<arr[index]> gets annotated as \<NonDet>.

Note that the Determinism Checker changes the type of an array access as described above if
the array access occurs on the rhs of an expression.
If it were to replace every array access with this rule, the checker would allow
invalid assignments to array elements. Example:
\begin{Verbatim}
@Det int @OrderNonDet [] x;
@Det int index;
@NonDet int y;
x[index] = y;
\end{Verbatim}
Here, we expect the checker to flag the assignment \code{x[i] = y} as an error.
The reason we expect this error is because assigning a value at a specific index
of an \<@OrderNonDet> array is equivalent to removing an element at that index
and assigning a new value which makes the resulting array \<@NonDet>. For instance,
consider an \<@OrderNonDet> array \<a> with elements \<[1,2,3]>. In one execution,
the statement \code{a[0] = 5} could result in the array \<[5,2,3]>. In another
execution, the same statement could result in the array \<[5,1,2]>.
Assigning values at a \<@NonDet> index in a \<@Det> array has a similar effect.
Had the checker replaced the type of every \<@OrderNonDet> array access with
\<@NonDet>, the array access \<x[i]> would have the type \<@NonDet> and
this assignment would not be flagged as an error.

\subsectionAndLabel{Valid user defined types}{determinism-valid-user-types}
A user defined type may be annotated as \<@OrderNonDet> if and only if it
is a subtype of Collection or Iterator.
For example, if a user defines a type as
\begin{Verbatim}
    public class TestUserCollection<E> extends ArrayList<E> {...}
\end{Verbatim}
Writing \code{@OrderNonDet TestUserCollection<@Det Integer>} is valid.\\
Writing \code{@OrderNonDet TestUserCollection<@NonDet Integer>} is invalid
because the type of the type parameter (\code{@NonDet}) is not a subtype
of the user defined Collection type (\code{@OrderNonDet}).

User defined types that are not a subtype of Collection or Iterator
may not be annotated as \code{@OrderNonDet}.
%Formally, the type validity rule is stated as follows:

%\begin{prooftree}
%  \AxiomC{is\_Collection\_Type(tc)}
%  \AxiomC{outer\_type(tx) <= tc}
%  \BinaryInfC{|- : tc<tx> }
%\end{prooftree}

% \begin{verbatim}
% Type outer_type( Type tx )
% {
%     if ( ! is_Generic_type(tx) )
%         return tx
%     Assert (tx == ty<tz>)
%         return ty
% }
% \end{verbatim}

\sectionAndLabel{Collection invariance}{determinism-invariance}
The Determinism Checker treats Collection classes as
invariant with respect to the determinism qualifier. That is, a
\code{@NonDet Collection} is unrelated by subtyping to an \code{@OrderNonDet Collection}
which is unrelated to a \code{@Det Collection}.

Assume for the sake of contradiction that collections are not invariant w.r.t. type qualifiers.
That would make the following code type-check, even though it
non-deterministically changes a \code{@Det List}:
\begin{Verbatim}
  void test(@Det List<@Det String> dList, @NonDet List<@Det String> nList) {
    @NonDet List<@Det String> ndList = dList;
    ndList.addAll(nList);
  }
\end{Verbatim}

The above code is unsafe because \<nList> and \<dList> are aliases to
the same \<List> object and \<nList.addAll(nList)> mutates the \<@Det>
reference \<dList> non-deterministically.

\sectionAndLabel{Collection aliasing and safe mutation}{determinism-aliasing-mutation}
It is possible to create a \<@NonDet> alias to an \<@OrderNonDet> or a \<@Det> collection.
For instance, calling \<next> or \<get> on an \code{@OrderNonDet Collection} creates a
\<@NonDet> alias.
Consider the example below:
\begin{Verbatim}
  void aliasTest(@OrderNonDet Set<@Det List<@Det String>> set, @NonDet int index, @Det String str) {
    @NonDet List<@Det String> lst = set.iterator().next();
    lst.add(index, str);
  }
\end{Verbatim}

Variable \<lst> has type \code{@NonDet List<@Det String>} (as a result of \<Set> iteration)
but an alias has type \code{@Det List<@Det String>} (as a member of \<Set>).
The call \<lst.add(...)> must not type-check because it mutates \<lst> non-deterministically,
thereby violating the determinism guarantees provided by the \<@Det> reference.

In order to prevent such unsafe behavior, the Determinism Checker must
prevent calls to mutation methods (like \<lst.add> above) on receivers of
type \code{@NonDet Collection<@OrderNonDet E>} or \code{@NonDet Collection<@Det E>}.
It achieves this via a method declaration annotation
\refqualclass{checker/determinism/qual}{CheckReceiverForMutation}.
The receiver of a method with this annotation must satisfy the following property: if the
receiver qualifier type is \<@NonDet>, then its type argument must be \<@NonDet>.
We annotate \<List.add> method in the JDK with \<@CheckReceiverForMutation>.
As a result, the checker reports an error at \code{lst.add(index, str);}
since the receiver qualifier has type \<@NonDet> and its type argument is not \<@NonDet>.

\subsectionAndLabel{HashSet and HashMap}{determinism-hash-set}
The \code{HashSet} and \code{HashMap} classes are special. Because iteration order is
non-deterministic, it is impossible to construct a \code{@Det HashSet}
or \code{@Det HashMap}.
Accordingly, the return type of the \code{HashSet} and \code{HashMap}
constructors are annotated as \code{@PolyDet("upDet")}. See
Section~\ref{determinism-polymorphism-ordernondet}.

\sectionAndLabel{Controlling polymorphism}{determinism-polymorphism}

\begin{description}
\item[\refqualclass{checker/determinism/qual}{PolyDet}] is a
  polymorphic qualifier over nondeterminism (see Section~\ref{qualifier-polymorphism}).
\end{description}

The Determinism Checker makes
several enhancements to standard polymorphic qualifiers.


\subsectionAndLabel{@PolyDet Variants for Special Cases}{determinism-polymorphism-ordernondet}

\begin{description}
\item[\refqualclass{checker/determinism/qual}{PolyDet}\<("down")>]
  Acts like \<@Det> when the polymorphic qualifier is instantiated as
  \<@OrderNonDet>.
  ``\<down>'' refers to the direction in the qualifier hierarchy:
  \<@OrderNonDet> is changed to a lower qualifier that is a subtype of it.
\item[\refqualclass{checker/determinism/qual}{PolyDet}\<("up")>]
  Acts like \<@NonDet> when the polymorphic qualifier is instantiated as
  \<@OrderNonDet>.
  ``\<up>'' refers to the direction in the qualifier hierarchy:
  \<@OrderNonDet> is changed to a higher qualifier that is a supertype of it.
\item[\refqualclass{checker/determinism/qual}{PolyDet}\<("upDet")>]
  Acts like \<@OrderNonDet> when the polymorphic qualifier is instantiated as
  \<@Det>.
  ``\<up>'' refers to the direction in the qualifier hierarchy and ``\<Det>''
  refers to the annotation that is modified:
  \<@Det> is changed to a higher qualifier that is a supertype of it.
\end{description}

Ordinarily, every occurrence of \<@PolyDet> is instantiated to the same
type.  Sometimes this does not capture the desired behavior.  For instance,
there should be three versions of the \<List.add> and \<List.hashCode> routines:

\begin{Verbatim}
         @NonDet boolean add(      @NonDet List<E> this, E e)
            @Det boolean add( @OrderNonDet List<E> this, E e)
            @Det boolean add(         @Det List<E> this, E e)

         @NonDet int hashCode(     @NonDet List<E> this)
         @NonDet int hashCode(@OrderNonDet List<E> this)
            @Det int hashCode(        @Det List<E> this)
\end{Verbatim}

A standard polymorphic annotation

\begin{Verbatim}
        @PolyDet boolean add(     @PolyDet List<E> this, E e)
        @PolyDet int hashCode(    @PolyDet List<E> this)
\end{Verbatim}

\noindent
can express the first and third versions of each routine, but not the
second one.  The \refqualclass{checker/determinism/qual}{PolyDet}\<("down")> and
\refqualclass{checker/determinism/qual}{PolyDet}\<("up")> annotations enable
expressing the method behavior:

\begin{Verbatim}
@PolyDet("down") boolean add(     @PolyDet List<E> this, E e)
  @PolyDet("up") int hashCode(    @PolyDet List<E> this)
\end{Verbatim}

For an example of \refqualclass{checker/determinism/qual}{PolyDet}\<("upDet")>,
consider the three versions of the \<HashSet> constructor that uses the elements
of another \<Collection>:

\begin{Verbatim}
         @NonDet HashSet(      @NonDet Collection<? extends E> c)
    @OrderNonDet HashSet( @OrderNonDet Collection<? extends E> c)
    @OrderNonDet HashSet(         @Det Collection<? extends E> c)
\end{Verbatim}

Since a \<HashSet> can never be deterministic, the return type in the third case
must be \<@OrderNonDet>. This is exactly what
\refqualclass{checker/determinism/qual}{PolyDet}\<("upDet")> expresses.

See the annotated JDK for more example uses of the annotations.

Since the type \<@OrderNonDet> is invalid for non-collections,
if \<@PolyDet> is applied to a collection somewhere in the method signature
and if the return type isn't a collection, then the return type should be
annotated as \<@PolyDet("up")> or \<@PolyDet("down")>.

\subsubsectionAndLabel{Special handling of Set.equals()}{determinism-improved-precision-set-equals}

In the JDK, the \<equals()> method of \<Set> is annotated as
\begin{Verbatim}
@PolyDet("up") boolean equals(@PolyDet Set<E> this, @PolyDet Object o);
\end{Verbatim}
While this is sound, it is imprecise.
For example, consider the code below:

\begin{Verbatim}
    @OrderNonDet Set<@Det Integer> a;
    @OrderNonDet Set<@Det Integer> b;
    a.equals(b);
\end{Verbatim}

Here, the type of \<a.equals(b)> is \<@NonDet>,
but it is deterministic, so it could be more precisely annotated as \<@Det>.
To achieve this precision, the checker annotates the type
of \<equals()> invoked on a receiver of type \<Set> as \<@Det> if the
following conditions are true:
\begin{itemize}
    \item the receiver does not have \<List> or its subtype as a type argument,
    \item the argument to equals() is also an \<@OrderNonDet Set>, and
    \item the argument to equals() also does not have \<List> or its subtype as a
    type parameter
\end{itemize}

\subsectionAndLabel{Distinguishing bindings and uses}{determinism-polymorphism-binding}

\begin{description}
\item[\refqualclass{checker/determinism/qual}{PolyDet}\<("use")>]
  Has a type qualifier that is a subtype of \<@PolyDet> qualifiers in the method
  declaration, but does not affect their instantiation.
\end{description}

As explained in Section~\ref{qualifier-polymorphism},
all of the \<@PolyDet> annotations in a method declaration are given the same
instantiation, which is one that works for all of the actual arguments
(their least upper bound).

The \<@PolyDet("use")> annotation is given an instantiation that is a subtype of all of the
\<@PolyDet> annotations, but does not affect the instantiation. This is especially
useful in preventing methods from non-deterministically modifying the
state of a deterministic receiver.

As an example of the need for the \<@PolyDet("use")> annotation, consider the
\<List.set> method.  There should be two versions of it (\<@OrderNonDet
int> is not a valid type, so no \<@OrderNonDet> instantiation is created):

\begin{Verbatim}
E set(@NonDet List<E> this, @NonDet int index, E element)
E set(   @Det List<E> this,    @Det int index, E element)
\end{Verbatim}

\noindent
However, it should \emph{not} be legal for a client to make a call such as
the following:

\begin{Verbatim}
@Det List<...> myDetList;
@NonDet int random;
myDetList.set(random, ...);
\end{Verbatim}

\noindent
The type system must forbid this call, because it modifies a
\<@Det> list in a nondeterministic way.
However, the arguments are legal for the first instantiation:
each argument type is a subtype of the corresponding declared type.

To solve this problem, the instantiation is chosen based only on the first
argument (the \<List> argument).  To have this effect, the library method
is annotated as:

\begin{Verbatim}
E set(@PolyDet List<E> this, @PolyDet("use") int index, E element)
\end{Verbatim}

As alluded to earlier, \<@PolyDet("use")> is resolved as a
separate polymorphic qualifier. If that qualifier is a supertype
of the instantiation of \<@PolyDet>, then the instantiation of \<@PolyDet>
is used as the instantiation of \<@PolyDet("use")>.
To illustrate how this helps precision,
consider the \<List.addAll> method. There should be six versions of it:

\begin{Verbatim}
@NonDet boolean addAll(@NonDet List<E> this, @NonDet Collection<? extends E> c)
@NonDet boolean addAll(@NonDet List<E> this, @OrderNonDet Collection<? extends E> c)
@NonDet boolean addAll(@NonDet List<E> this, @Det Collection<? extends E> c)
@Det    boolean addAll(@OrderNonDet List<E> this, @OrderNonDet Collection<? extends E> c)
@Det    boolean addAll(@OrderNonDet List<E> this, @Det Collection<? extends E> c)
@Det    boolean addAll(        @Det List<E> this, @Det Collection<? extends E> c)
\end{Verbatim}

To allow all six instantiations shown above, the library method
is annotated as:

\begin{Verbatim}
@PolyDet("down") boolean addAll(@PolyDet List<E> this, @PolyDet("use") Collection<? extends E> c)
\end{Verbatim}
We treat collections as invariant with respect to determinism type qualifiers,
a \<@Det> collection is unrelated an \<@OrderNonDet> collection, which is unrelated
to a \<@Det> collection. Therefore, \<@PolyDet("use")> must allow qualifiers that
are subtypes of the qualifiers that \<@PolyDet> resolves to in order to allow
instantiations like \code{@Det boolean addAll(@OrderNonDet List<E> this, @Det Collection<? extends E> c)}.

Whenever \<@PolyDet("up")>, \<@PolyDet("down")>, and \<@PolyDet("upDet")> are written on formal parameters,
they are treated as \<@PolyDet("use")> sites that are instantiated with the corresponding \<@PolyDet> mappings.
For instance, consider the following method signature:
\begin{Verbatim}
static void myMethod(@PolyDet List<E> lst, @PolyDet("up") int index);
\end{Verbatim}
Only the following calls are legal:
\begin{Verbatim}
@NonDet List<...> ndList; @NonDet int index;
myMethod(ndList , index);

@NonDet List<...> ndList; @Det int index;
myMethod(ndList , index);

@OrderNonDet List<...> ondList; @NonDet int index;
myMethod(ondList , index);

@OrderNonDet List<...> ondList; @Det int index;
myMethod(ondList , index);

@Det List<...> detList; @Det int index;
myMethod(detList , index);
\end{Verbatim}

\subsectionAndLabel{Disallowing \<@OrderNonDet> in polymorphic instantiations}{determinism-no-ordernondet}
\begin{description}
    \item[\refqualclass{checker/determinism/qual}{PolyDet}\<("noOrderNonDet")>] is the same as \<@PolyDet> except that it cannot be instantiated with \<@OrderNonDet>.
\end{description}
The method signature:
\begin{Verbatim}
static void myMethodNoOrderNonDet(@PolyDet("noOrderNonDet") List<E> lst);
\end{Verbatim}
allows the following calls:
\begin{Verbatim}
@NonDet List<...> ndList;
myMethodNoOrderNonDet(ndList);

@Det List<...> detList;
myMethodNoOrderNonDet(detList);
\end{Verbatim}
The call \code{myMethodNoOrderNonDet(ondList);} where \<ondList>  is of type \code{@OrderNonDet List<...>} is illegal.
In the JDK, we annotate the \<remove> method of \<List> as follows:
\begin{Verbatim}
@PolyDet("up") E remove(@PolyDet("noOrderNonDet") List<E> this, @PolyDet("use") int index);
\end{Verbatim}
This method signature disallows removing elements at specific indices from an  \<@OrderNonDet List>.
Allowing such removals would be unsound. Consider the following example:
\begin{Verbatim}
@OrderNonDet List<@Det Integer> ondList;         // Suppose ondList contains elements: [1, 2, 3]
ondList.remove(0);
\end{Verbatim}
Since the list is \<@OrderNonDet>, removing an element at index 0 could remove any of its 3 elements in different runs
making the resulting list \<@NonDet> which is unsound.

Similar to \<@PolyDet("use")>, \<@PolyDet("useNoOrderNonDet")>
annotation is given an instantiation that is a subtype of the
\<@PolyDet("noOrderNonDet")> annotations, but does not affect the instantiation.

\subsectionAndLabel{Type qualifier hierarchy}{determinism-hierarchy}
\begin{figure}
    \begin{center}
        \includeimage{determinism-poly-hierarchy}{6cm}
    \end{center}
    \caption{\<@PolyDet> type hierarchy}
    \label{fig-determinism-poly-hierarchy}
\end{figure}
Figure~\ref{fig-determinism-poly-hierarchy} shows the type qualifier hierarchy among all the
variants of \<@PolyDet>. Notice that \<@PolyDet> and \<@PolyDet("use")> are considered
to be same in this hierarchy.
At method call sites, \<@PolyDet("use")> gets replaced by the type qualifier that \<@PolyDet>
resolves to.

\begin{figure}
    \begin{center}
        \includeimage{determinism-all-hierarchy}{8cm}
    \end{center}
    \caption{Determinism type hierarchy}
    \label{fig-determinism-all-hierarchy}
\end{figure}
Figure~\ref{fig-determinism-all-hierarchy} shows the type qualifier hierarchy among all the
type qualifiers in the Determinism Checker.

\sectionAndLabel{How to customize the Determinism Checker}{determinism-customization}

In the JDK, printing methods such as \<println> are annotated as taking a
deterministic result.  This guarantees that a program's output will be
deterministic.  If it is too restrictive for your program:
\begin{itemize}
\item
  You can define your own printing methods, some of which can take
  nondeterministic inputs.
\item
  You can override the JDK annotations by providing a stub file
  (Section~\ref{stub}).
\end{itemize}


\sectionAndLabel{Determinism of formatting routines}{determinism-formatting-routines}

Suppose you want to ensure that your program's printed output is
deterministic.
It is \emph{not} enough to annotate the formal parameters of \<format> as
\<@Det>, because the arguments' class may not have a \<toString> implementation that is deterministic.
The \<Object.toString> method is nondeterministic, so it is annotated as
\code{@NonDet String toString(@PolyDet Object this)}.  Most subclasses
that override \<toString> do so determiniscially:
\code{@PolyDet String toString(@PolyDet MyClass this)}.

The output of \<format> is deterministic only if its arguments are
deterministic and their types have a deterministic \<toString> implementation.
The method declaration annotation
\refqualclass{checker/determinism/qual}{@RequiresDetToString} expresses the
latter requirement.

As an illustration, consider the following method annotated with \<@RequiresDetToString>:
\begin{Verbatim}
    @RequiresDetToString
    @PolyDet String n(@PolyDet NewClass o) {...}
\end{Verbatim}
Unless \<NewClass> has a \<toString> implementation that returns \<@Det> or \<@PolyDet>,
the checker will report an error.

A user can also specify which specific arguments the check must apply to by
supplying annotation values to \<RequiresDetToString>.
For example, if the user annotates a method as:
\begin{Verbatim}
    @RequiresDetToString{0, 2}
    @PolyDet String n(@PolyDet NewClass1 o1, @PolyDet NewClass2 o2, @PolyDet NewClass3 o3) {...}
\end{Verbatim}
then the checker will enforce deterministic \<toString> checks on \<NewClass1> and \<NewClass3>,
but not on \<NewClass2>.


\sectionAndLabel{Default annotations}{determinism-defaults}

When reading bytecode (such as libraries that you do not compile from
source), the default annotation is \<@PolyDet>.

In source code, the following locations default to
\<@Det>, unless command-line option \<-AusePolyDefault> is supplied in
which case they default to \<@PolyDet>:
\begin{itemize}
    \item Formal parameter and return types. \\
      As an exception, the return type of a method with no unannotated or
    \<@PolyDet> formal parameters (including the receiver) defaults to
    \<@Det>, regardless of whether \<-AusePolyDefault> was supplied.
    \item Constructor results
    \item Component types of array parameter/return types
\end{itemize}

The default type qualifier for instance fields in classes annotated with
\refqualclass{framework/qual}{HasQualifierParameter} is \<@PolyDet>.
The default for other fields, including all static fields, is \<@Det>.

%These choices permit all possible arguments to an unannotated method or
%constructor.

\subsectionAndLabel{Local array and generic types}{determinism-array-defaults}
According to the CLIMB to top rule (see Section~\ref{climb-to-top}), the default
for a local collection or array would be a \code{@NonDet} collection or array
with \code{@Det} elements. For example, in the following local declarations
\begin{Verbatim}
    List<String> list;
    String[] arr;
\end{Verbatim}
the type of \code{list} would be \code{@NonDet List<@Det String>} and
\code{arr} would be \code{@Det String @NonDet []}. These types are invalid (see
Section~\ref{determinism-hash-set}). For \code{Lists}, the Determinism Checker
instead makes the default qualifier for local variables \code{@Det}. Also, if a local
\code{List} variable is unannotated, then any unannotated component types are
defaulted to \code{@Det} as well. For other collections and arrays, the default
for the collection or array and all its components is \code{@NonDet}. This means
that local collection and array declarations are valid by default.

With the Determinism Checker's Rules, the type of \code{list} above is actually
\code{@Det List<@Det String>} and the type of \code{arr} is \code{@NonDet
String @NonDet []}, which are valid types.

\sectionAndLabel{Implicit annotations}{determinism-implicits}
The parameter type of the \code{main} method has an implicit annotation
\<@Det String @Det[]>.


\sectionAndLabel{Utility routines for arrays and collections}{determinism-utilities}

Class \refclass{checker/determinism}{DeterminismUtil} provides static
utility methods for working with arrays and collections.  They are properly
annotated with determinism type qualifiers, so using them may reduce the number
of annotations that you need to write.


%%  LocalWords:  DeterminismChecker HashSet InputFile NonDet Det MyClass
%%  LocalWords:  OrderNonDet PolyDet UsePolyDet println MyType keySet rhs
%%  LocalWords:  CollectionType TestUserCollection upDet ordernondet addAll
%%  LocalWords:  noOrderNonDet myMethodNoOrderNonDet ondList implicits
%%  LocalWords:  useNoOrderNonDet HasQualifierParameter boolean
