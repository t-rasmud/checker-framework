\htmlhr
\chapter{GUI Effect Checker\label{guieffect-checker}}

One of the most prevalent GUI-related bugs is \emph{invalid UI update} or \emph{invalid thread access}:  accessing the UI directly from a
background thread.


% For performance reasons, most applications with a graphical user interface (GUI) use multiple
% threads.

Most GUI frameworks (including Android, AWT, Swing, and SWT) create
a single distinguished thread --- the UI event thread
--- that handles all GUI events and updates.
% Every UI event (mouse click, key press,
% etc.) is dispatched to that thread.
To keep the interface responsive, any expensive computation should be
offloaded to \emph{background threads} (also called \emph{worker threads}).
If a background thread accesses a UI
element such as a JPanel (by calling a JPanel method or reading/writing a
field of JPanel),
the GUI framework raises an exception that terminates the program.
To fix the bug, the background thread should send a request to the
UI thread to perform the access on its behalf.

It is difficult for a programmer to remember which methods may be called on
which thread(s).
The GUI Effect Checker solves this problem.
The programmer annotates each method to indicate whether:
\begin{itemize}
\item
  It accesses no UI elements (and may run on any thread);
  such a method is said to have the ``safe effect''.
\item
  It may access UI elements (and must run on the UI thread);
  such a method is said to have the ``UI effect''.
\end{itemize}


%% True and interesting, but not relevant to people who are reading this
%% section of the manual.
% The GUI Effect Checker prevents this problem by using an \emph{effect
% system}, also known as a type-and-effect system.
% An \emph{effect} is a property of a \emph{computation}.
% A well-known example of an effect system is Java's checked exception mechanism.  When Java's
% type system says a method may throw exceptions \code{A} or \code{B} (the method type-checks with a
% \code{throws A, B} clause), this makes two important statements about the behavior of the method at
% run time.  First, running the method in question may result in one of the specified exceptions being
% thrown.  Second, if an exception is thrown by the method, then the exception must be an \code{A} or
% \code{B}.  So an effect for a method (or any other block of code) is an upper bound on a certain
% class of behaviors.  In general, a block of code with a given effect (upper bound on behavior) may
% call methods with the same, or weaker effects (e.g., no throws clause), because this will not
% violate the assumed upper bound.  Calling a method with an incompatible effect (e.g., calling a
% \code{throws C} method) would be a compile-time error.  For reasoning about effects to be sound, the
% effects must interact with inheritance: method overrides must have the same (or lower) upper bound
% on behavior as the method being overridden.  For checked exceptions, this means the method override
% cannot throw any exceptions not permitted by the parent class implementation's \code{throws} clause.

The GUI Effect Checker verifies these effects and statically enforces that UI methods are only
called from the correct thread.  A method with the safe effect is prohibited from calling a method
with the UI effect.
% All UI element methods have the ``UI effect.''

For example, the effect system can reason about when method calls must be dispatched to the UI
thread via a message such as \<Display.syncExec>.
\begin{alltt}
@SafeEffect
public void calledFromBackgroundThreads(JLabel l) \{
    l.setText("Foo");       // Error: calling a @UIEffect method from a @SafeEffect method
    Display.syncExec(new Runnable \{
        @UIEffect // inferred by default
        public void run() \{
            l.setText("Bar");  // OK: accessing JLabel from code run on the UI thread
        \}
    \});

\}
\end{alltt}

%% True, but why tip off readers that this is complex?  They might read it
%% and just understand it...
% The bulk of the GUI Effect Checker's complexity comes from handling interactions between effects and
% inheritance, and from \emph{effect polymorphism} (code that can work as either safe or UI code).

The GUI Effect Checker's annotations fall into three categories:

\begin{itemize}
\item
  effect annotations on methods (Section~\ref{guieffect-annotations}),
\item
 class or package annotations controlling the default effect (Section~\ref{guieffect-defaults}), and
\item
  \emph{effect-polymorphism}:  code that works for both the safe effect and
  the UI effect (Section~\ref{guieffect-polymorphism}).
\end{itemize}


\section{GUI effect annotations\label{guieffect-annotations}}

There are two primary GUI effect annotations:
\begin{itemize}
\item
  \refqualclass{checker/guieffect/qual}{SafeEffect}
  is a method annotation marking code that must not
  access UI objects.
\item
  \refqualclass{checker/guieffect/qual}{UIEffect}
  is a method annotation marking code that may access
  UI objects.  Most UI object methods (e.g., methods of \<JPanel>) are
  annotated as \code{@UIEffect}.
\end{itemize}

\code{@SafeEffect} is a sub-effect of \code{@UIEffect}, in that it is always safe to
call a \code{@SafeEffect} method anywhere it is permitted to call a
\code{@UIEffect} method.  We write this relationship as

\centerline{\code{@SafeEffect} $\prec$ \code{@UIEffect}}


\section{What the GUI Effect Checker checks\label{guieffect-checks}}

The GUI Effect Checker ensures that only the UI thread accesses UI objects.
This prevents GUI errors such
as invalid UI update and invalid thread access.

The GUI Effect Checker issues errors in the following cases:

%% TODO: Is this exhaustive?  Read the code to find out.

\begin{itemize}
\item
  A \<@UIEffect> method is invoked by a \<@SafeEffect> method.

\item
  Method declarations violate subtyping restrictions:  a supertype declares
  a \code{@SafeEffect} method, and a subtype annotates an overriding
  version as \code{@UIEffect}.

\end{itemize}

Additionally, if a method implements or overrides a method in two
supertypes (two interfaces, or an interface and parent class), and those
supertypes give different effects for the methods, the GUI Effect Checker
issues a warning (not an error).



% It checks this by attaching to each method in the program an effect, which is an upper bound on
% which objects it may touch: the \code{@UIEffect} effect (and method annotation) marks code that may
% access UI elements, and therefore must execute only on the distinguished UI event thread; the
% \code{@SafeEffect} effect and method annotation marks code that must not directly access UI
% elements, and is therefore safe to call from any thread.
%
% Polymorphic effects are used for classes and interfaces which are used in both ways, for example the
% \code{java.util.Runnable} interface, containing one method, which is used for tasks including
% starting new background threads (in which case the code must have the \code{@SafeEffect}) and
% dispatching code to run on the UI thread (in which case the code may have the \code{@UIEffect}).


\section{Running the GUI Effect Checker\label{guieffect-running}}

The GUI Effect Checker can be invoked by running the following command:
\begin{Verbatim}
  javac -processor org.checkerframework.checker.guieffect.GuiEffectChecker MyFile.java ...
\end{Verbatim}


\section{Annotation defaults\label{guieffect-defaults}}

The default method annotation is \code{@SafeEffect}, since most code in most programs is not related
to the UI\@.  This also means that typically, code that is unrelated to the UI need not be annotated
at all.

The GUI Effect Checker provides three primary ways to change the default method effect for a class
or package:
\begin{itemize}
\item
  \refqualclass{checker/guieffect/qual}{UIType}
  is a class annotation that makes the effect for unannotated methods in that
class default to
\code{@UIEffect}.  (See also \code{@UI} in Section~\ref{guieffect-polymorphism-using}.)
\item
  \refqualclass{checker/guieffect/qual}{UIPackage}
  is a \emph{package} annotation, that makes the effect for unannotated
methods in that package default to \code{@UIEffect}.  It is not transitive; a package nested inside
a package marked \code{@UIPackage} does not inherit the changed default.
\item
  \refqualclass{checker/guieffect/qual}{SafeType} is a class annotation that makes the effect for unannotated methods in that
class default to \code{@SafeEffect}.  Because \code{@SafeEffect} is already the default
effect, \code{@SafeType} is only useful for class types inside a package marked \code{@UIPackage}.
\end{itemize}

There is one other place where the default annotation is not automatically \code{@SafeEffect}:
anonymous inner classes.  Since anonymous inner classes exist primarily for brevity, it would be
unfortunate to spoil that brevity with extra annotations.  By default, an anonymous inner class
method that overrides or implements a method of the parent type inherits that method's effect.
For example, an anonymous inner class implementing an interface with method \code{@UIEffect void
m()} need not explicitly annotate its implementation of \code{m()}; the implementation will inherit
the parent's effect.  Methods of the anonymous inner class that are not inherited from a parent type
follow the standard defaulting rules.


\section{Polymorphic effects\label{guieffect-polymorphism}}

Sometimes a type is reused for both UI-specific and background-thread work.  A good example is the
\code{Runnable} interface, which is used both for creating new background threads (in which case the
\code{run()} method must have the \code{@SafeEffect}) and for sending code to the UI thread to
execute (in which case the \code{run()} method may have the
\code{@UIEffect}).  But the declaration of
\code{Runnable.run()} may have only one effect annotation in the source code.
How do we reconcile these
conflicting use cases?

\emph{Effect-polymorphism} permits a type to be used for both UI and non-UI
purposes.  It is similar to Java's generics in that you define, then use, the
effect-polymorphic type.
Recall that to \emph{define} a generic type, you write a type parameter such as
\code{<T>} and use it in the body of the type definition; for example,
 \code{class List<T> \ttlcb\ ...\ \ T get() \ttlcb...\ttrcb\ \ ...\ \ \ttrcb}.
To \emph{instantiate} a generic type, you write its name along with a type
argument; for example, \code{List<Date> myDates;}.


\subsection{Defining an effect-polymorphic type\label{guieffect-polymorphism-defining}}

To declare that a class is effect-polymorphic, annotate its definition with
\refqualclass{checker/guieffect/qual}{PolyUIType}.
To use the effect variable in the class body, annotate a method with \refqualclass{checker/guieffect/qual}{PolyUIEffect}.
It is an error to use \code{@PolyUIEffect} in a class that is not
effect-polymorphic.


Consider the following example:

\begin{alltt}
@PolyUIType
public interface Runnable \{
    @PolyUIEffect
    void run();
\}
\end{alltt}

\noindent
This declares that class \<Runnable> is parameterized over one
generic effect, and that when \<Runnable> is instantiated, the effect
argument will be used as the effect for the \<run> method.


\subsection{Using an effect-polymorphic type\label{guieffect-polymorphism-using}}

To instantiate an effect-polymorphic type, write one of these three type
qualifiers before a use of the type:
\begin{itemize}
\item
  \refqualclass{checker/guieffect/qual}{AlwaysSafe}
  instantiates the type's effect to \code{@SafeEffect}.
\item
  \refqualclass{checker/guieffect/qual}{UI}
  instantiates the type's effect to \code{@UIEffect}.
  \emph{Additionally}, it changes the
  default method effect for the class to \code{@UIEffect}.
\item
  \refqualclass{checker/guieffect/qual}{PolyUI} instantiates the type's
  effect to \code{@PolyUIEffect} for the same instantiation as the current
  (containing) class.  For example, this is the qualifier of the receiver
  \code{this} inside a method of a \code{@PolyUIType} class, which is how
  one method of an effect-polymorphic class may call an effect-polymorphic
  method of the same class.
\end{itemize}

As an example:

\begin{alltt}
@AlwaysSafe Runnable s = ...;    s.run();    // s.run() is @SafeEffect
@PolyUI Runnable p = ...;        p.run();    // p.run() is @PolyUIEffect (context-dependent)
@UI Runnable u = ...;            u.run();    // u.run() is @UIEffect
\end{alltt}

It is an error to apply an effect instantiation qualifier to a type that is not effect-polymorphic.

Note that no annotation is required on the anonymous class declaration itself (e.g. \code{new Runnable()\{...\}}
does not require a type use annotation, although the variable, field, or argument it ends up being assigned to might).
Instead, the GUI Effect Checker will infer the effect qualifier based on the method being called from within the
members of that specific anonymous class.

\subsection{Subclassing a specific instantiation of an effect-polymorphic type\label{guieffect-subclassing}}

Sometimes you may wish to subclass a specific instantiation of an effect-polymorphic type, just as
you may extend \code{List<String>}.

To do this, simply place the effect instantiation qualifier by the name of the type you are
defining, e.g.:

\begin{alltt}
@UI
public class UIRunnable extends Runnable \{...\}
@AlwaysSafe
public class SafeRunnable extends Runnable \{...\}
\end{alltt}
The GUI Effect Checker will automatically apply the qualifier to all classes and interfaces the class
being defined extends or implements.  (This means you cannot write a class that is a subtype of a
\code{@AlwaysSafe Foo} and a \code{@UI Bar}, but this has not been a problem in our experience.)


\subsection{Subtyping with polymorphic effects\label{guieffect-subtyping}}

With three effect annotations, we must extend the static sub-effecting relationship:

\centerline{\code{@SafeEffect} $\prec$ \code{@PolyUIEffect} $\prec$ \code{@UIEffect}}

\noindent
This is the correct sub-effecting relation because it is always safe to
call a \code{@SafeEffect}
method (whether from an effect-polymorphic method or a UI method), and a \code{@UIEffect} method
may safely call any other method.
% (since the instantiation of \code{@PolyUIEffect} may be at most
% \code{@UIEffect} itself).

This induces a subtyping hierarchy on type qualifiers:

\centerline{\code{@AlwaysSafe} $\prec$ \code{@PolyUI} $\prec$ \code{@UI}}

\noindent
This is sound because a method instantiated according to any qualifier will always be
safe to call in place of a method instantiated according to one of its super-qualifiers.
This allows clients to pass ``safer'' instances of some object type to a given method.


\subsubsection{Effect polymorphism and arguments\label{guieffect-overrides}}

Sometimes it is useful to have \code{@PolyUI} parameters on a method.  As a trivial example, this
permits us to specify an identity method that works for both \code{@UI Runnable} and
\code{@AlwaysSafe Runnable}:

\begin{alltt}
public @PolyUI Runnable id(@PolyUI Runnable r) \{
    return r;
\}
\end{alltt}

\noindent
This use of \code{@PolyUI} will be handled as is standard for polymorphic qualifiers in the Checker
Framework (see Section \ref{qualifier-polymorphism}).

\code{@PolyUIEffect} methods should generally not use \code{@PolyUI} arguments: it is permitted by
the framework, but its interaction with inheritance is subtle, and may not behave as you would
hope.

The shortest explanation is this: \code{@PolyUI} arguments may only be overridden by \code{@PolyUI}
arguments, even though the implicitly \code{@PolyUI} \emph{receiver} may be overridden with a
\code{@AlwaysSafe} receiver.

As noted earlier (Section \ref{covariant-type-parameters}), Java's generics are invariant ---
\code{A<X>} is a subtype of \code{B<Y>} only if \code{X} is identical to \code{Y}.
Class-level use of \code{@PolyUI} behaves slightly differently.
Marking a type declaration \code{@PolyUIType} is conceptually
equivalent to parameterizing the type by some \code{E extends Effect}.  But in this view,
\code{Runnable<SafeEffect>} (really \code{@AlwaysSafe Runnable}) would be considered a subtype of
\code{Runnable<UIEffect>} (really \code{@UI Runnable}), as explained earlier in this section.  Java's
generics do not permit this, which is called \emph{covariant} subtyping in the effect parameter.
Permitting it for all generics leads to problems where a type system can miss errors.  Java solves
this by making all generics invariant, which rejects more programs than strictly necessary, but
leads to an easy-to-explain limitation.  For this checker, covariant subtyping of effect parameters
is very important: being able to pass an \code{@AlwaysSafe
Runnable} in place of a \code{@UI Runnable} is extremely useful.  Since we need to allow some
cases for flexibility, but need to reject other cases to avoid missing errors, the distinction is a
bit more subtle for this checker.

Consider this small example (warning: the following is rejected by the GUI Effect Checker):

\begin{alltt}
@PolyUIType
public interface Dispatcher \{
    @PolyUIEffect
    void dispatch(@PolyUI Runnable toRun);
\}
@SafeType
public class SafeDispatcher implements Dispatcher \{
    @Override
    public void dispatch(@AlwaysSafe Runnable toRun) \{
        runOnBackgroundThread(toRun);
    \}
\}
\end{alltt}

This may initially seem like harmless code to write, which simply specializes the implicit effect
parameter from \code{Dispatcher} in the \code{SafeDispatcher} implementation.  However, the way
method effect polymorphism is implemented is by implicitly making the receiver of a
\code{@PolyUIEffect} method --- the object on which the method is invoked --- \code{@PolyUI}.  So
if the definitions above were permitted, the following client code would be possible:

\begin{alltt}
@AlwaysSafe SafeDispatcher s = ...;
@UI Runnable uitask = ...;
s.dispatch(uitask);
\end{alltt}

At the call to \code{dispatch}, the Checker Framework is free to consider \code{s} as its
supertype, \code{@UI SafeDispatcher}.  This permits the framework to choose the same qualifier for
both the (implicit) receiver use of \code{@PolyUI} and the \code{toRun} argument to
\code{Dispatcher.dispatch}, passing the checker.  But this code would then pass a UI-thread-only
task to a method which should only accept background thread tasks --- exactly what the checker
should prevent!

To resolve this, the GUI Effect Checker rejects the definitions above, specifically the
\code{@AlwaysSafe} on \code{SafeDispatcher.dispatch}'s parameter, which would need to remain
\code{@PolyUI}.

A subtlety of the code above is that the receiver for \code{SafeDispatcher.dispatch} is also
overridden, switching from a \code{@PolyUI} receiver to a \code{@Safe} receiver.  That change is
permissible.  But when that is done, the polymorphic arguments may no longer be interchangeable
with the receiver.


\section{References\label{guieffect-references}}

The ECOOP 2013 paper ``JavaUI: Effects for Controlling UI Object Access''
    includes some case
    studies on the checker's efficacy, including descriptions of the relatively few false warnings
    we encountered.
    It also contains a more formal description of the effect system.
    You can obtain the paper at: \\
  \myurl{https://homes.cs.washington.edu/~mernst/pubs/gui-thread-ecoop2013-abstract.html}


%  LocalWords:  SafeEffect calledFromBackgroundThreads JLabel setText AWT
%  LocalWords:  UIEffect syncExec Runnable UIType UIPackage SafeType SWT
%  LocalWords:  PolyUIType PolyUIEffect PolyUI UIRunnable SafeRunnable Foo
%  LocalWords:  JavaUI JPanel myDates AlwaysSafe toRun SafeDispatcher
%%  LocalWords:  runOnBackgroundThread uitask
